\chapter{Étude d'un prédicteur de Smith}
Nous allons maintenant passer à l'analyse d'un autre type de correction de systèmes à retard : le prédicteur de Smith. Ce type de correcteur permet d'établir une commande de système retardé en utilisant une estimation du procédé qui sera utilisé en temps $t$ et $t+h$.

\section{Schéma de principe du prédicteur de Smith}
Nous avons dessiné avec \emph{Simulink} le schéma bloc en figure \ref{fig:sch_predicteurSmith}, dans lequel nous avons séparé $\tilde{C}(p)$ (C\_Smith sur le schéma) et  $ G(p) e ^{-hp}$(Procédé dans le schéma).
\begin{figure}[!ht]
\centering
\includegraphics[width=\textwidth]{./IV/images/schema_Predicteur.pdf}
\caption{Schéma de principe du prédicteur de Smith}\label{fig:sch_predicteurSmith}
\end{figure}
Le bloc $C(p)$ devra contenir le bloc de commande que nous souhaitons utiliser. Il pourra être complexe ou bien plus simple.

\section{Fonction de transfert du système en boucle fermé}
Pour obtenir le transfert en boucle fermé de ce schéma, nous allons nous référencer à des résultats de cours et TD, qui sont : 
\begin{equation}
\tilde{C}(p) = \frac{C(p)}{1+G(p)\left(1-e^{-hp}\right)}C(p)	
\end{equation}
ainsi que le transfert de boucle fermé qui s'écrit :
\begin{align}
G_{bf} &= \frac{\tilde{C}G(p)e^{-hp}}{1+\tilde{C}G(p)e^{-hp}}\\
	   &= \frac{C(p)G(p)e^{-hp}}{1+G(p)C(p)}	
\end{align}

Nous utiliserons cette fonction de transfert pour déterminer, à partir du correcteur établi dans les prochaines parties, le polynôme caractéristique du système. 

\section{Correcteur sur le prédicteur de Smith}
On nous propose de mettre un correcteur simple dans la bloc $C(p)$, celui ci sera de a forme : $C(p) = k_0$, soit, un correcteur proportionnel. Nous obtenons avec ce nouvel élément, la fonction de transfert enboucle fermé suivante :
\begin{equation}
G_{bf}(p) = \frac{k_sk_rk_me^{-hp}}{\tau p +p+k_0k_sk_rk_m}
\end{equation}
L'analyse du polynôme caractéristique vient alors dans les équations suivantes :
\begin{align*}
\tau p + p + k_0k_sk_rk_m = 0 \Leftrightarrow p_{1,2} = \frac{-1 \pm \sqrt{1-4\tau k_0k_sk_rk_m}}{2\tau}
\end{align*}
Plusieurs conditions apparaissent alors sur le gain proportionnel pour assurer le cahier des charges : \begin{itemize}
\item [\textbullet]$Re(p_{1,2}) < 0$ pour assurer la stabilité. De cette hypothèse, nous avons alors forcement :\begin{equation}
1-4\tau k_0k_sk_rk_m < 1 \Leftrightarrow k_0 > 0
\end{equation}
Résultat logique, bien qu'il existe des correcteurs avec un gain admis de manière négative, nous sommes ici dans un cas trivial qui n'admet qu'un gain proportionnel positif.
\item [\textbullet] $1-4\tau k_0k_sk_rk_m \geq 0$ pour ne pas avoir des pôles complexes conjugués. Nous avons alors :
\begin{equation}
4\tau k_0k_sk_rk_m \leq 1 \Leftrightarrow k_0 \leq \frac{1}{4\tau k_sk_rk_m}
\end{equation}
\paragraph*{Valeur du Correcteur $C(p)$} \underline{Application numérique} Nous avons alors un correcteur qui est : \begin{equation}\label{eqn:corPredicteurSmith}
C(p) = 0.0925
\end{equation}
\end{itemize}

\section{Analyse stabilité, ajout d'une perturbation}
Nous avons prouvé, par construction du correcteur, que le système était asymptotiquement stable. Cependant, nous avons omis plusieurs hypothèse que nous allons analyser maintenant.

 
Il a été noté dans le cahier des charges l'ajout d'une perturbation sur la commande $u(t)$(voir schéma\ref{fig:sch_predicteurSmith}). Cette perturbation, de type échelon, ne doit pas déstabiliser le système. Nous allons étudier le transfert entre cette perturbation et la sortie du système. Cette étude a déjà été commencé au cours des cours/TD, elle est exprimé par \begin{equation}
\frac{Y(p)}{P_pert(p)}=\frac{G(p)e^{-hp}\left(1+G(p)\left(1-e^{-hp}\right)C(p)\right)}{1+G(p)\left(1-e^{-hp}\right)C(p)+G(p)e^{-hp}C(p)}
\end{equation}

\section{Simulation \emph{Matlab} et observation des résultats}
Nous avons pu simuler sur \emph{Matlab} la réponse à un échelon de position que vous trouverez en figure \ref{fig:rep_echeonSmith}.\begin{figure}[!ht]
\centering
\includegraphics[width = .7\textwidth]{./IV/images/rep_echelonSmith.pdf}
\caption{Réponse à un échelon unité en BF avec un Préditeur de Smith}\label{fig:rep_echeonSmith}
\end{figure}

Toute les attentes du cahier des charges ont été remplis pour cette partie, nous avons validé la simulation de cette commande.

\section{Changement du retard} 
Pour observer une partie de la robustesse du système, nous avons modifier radicalement la valeur du retard du système. Nous avons choisi une valeur hors des valeurs admises dans les partis de modélisation du procédé en boucle ouverte ou avec un correcteur Proportionnel dérivateur, pour voir si la stabilité du système était respecté. Nous avons obtenu la courbe \ref{fig:rep_echeonSmith5}. \begin{figure}[!ht]
\centering
\includegraphics[width = .7\textwidth]{./IV/images/rep_echelonSmith_h5.pdf}
\caption{Réponse à un échelon unité en BF avec un Préditeur de Smith et un retard $h=5s$}\label{fig:rep_echeonSmith5}
\end{figure}
Nous voyons que le système réponds toujours au cahier des charges, résultat logique car le transfert en boucle fermé n'a pas de quasi polynôme caractéristique