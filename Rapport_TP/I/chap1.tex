\chapter{Identification-Modélisation du système}
Dans un premier temps, nous allons déterminer les paramètres du moteur, ensuite, nous déterminerons le modèle fréquentiel ainsi que le modèle espace d'état du système. Puis, nous étudierons les propriétés, les performances et la stabilité du système. 
	\section{Détermination de paramètres et du retard}
	On identifiera les paramètres du moteur grâce à une approche dite \emph{boite noire}, c'est-à-dire que suivant la forme d'une réponse du système à un échelon, nous allons choisir une modélisation par fonction de transfert type (1\ier ordre, 2\ieme ordre, ...). Comme il s'agit d'un moteur à courant continu, nous choisissons un modèle du premier ordre car il permet de former un modèle de précision suffisante au vu de notre application.\\
Un modèle du 1\ier ordre est de la forme suivante :
\begin{equation}
G(p) = \frac{K}{\tau p+1}
\end{equation}
Où : 
\begin{description}
\item[$K$ :] Le gain statique du système.
\item[$\tau$ :] La constante de temps du système (en seconde).
\end{description}
Nous identifierons $K$ en mesurant le gain statique  de la réponse à un signal échelon (pour $t$ tel que la réponse se soit stabilisée): $ K = V_g(t)/U_m(t)$.


Pour l'estimation de $\tau$, nous utiliserons la relation suivante : $ \tau = t$ tel que $V_g(t)/U_m(t) = 0,63*K $.\\

Pour identifier le retard, que nous savons être sur la commande du moteur, nous allons étudier de déphasage entre un signal d'entrée de type rectangle à fréquence faible ($1$Hz) et la sortie $V_s(t)$. Nous savons analytiquement qu'un système du premier ordre à un déphasage nul à basse fréquence, donc à partir du déphasage mesuré nous pouvons obtenir le retard.

	\section{Autres méthode}
	\section{Modèle fréquentiel}
	\section{Modèle espace d'état}
	\section{Commandabilité et observabilité}
	\section{Analyse de la boucle ouverte}
	\section{Stabilité de la boucle fermée}
	Est-ce bien ces deux méthodes ? (la troisième méthode supposée étant le pseudo-retard non traité en cours)
		\subsection{Delay-Sweeping}
		\subsection{Stabilité 2D}


