\chapter{Étude d'une commande Proportionnelle-dérivateur}
\section{Intérêt de ce correcteur}
Pour établir notre asservissement en position, nous devons faire en sorte de commander le transfert entre $u_m$ et $V_s$. Ce transfert dispose d'un intégrateur pur et d'un pôle en $-\frac{1}{\tau _m}$, qui donnent l'instabilité de la position du moteur à une entrée échelon. Un premier correcteur nous est proposé sous la forme :
\begin{equation}\label{eqn:correcteurProportionnel}
C(p) = k_0(1+d_ip)
\end{equation} 
avec $k_0$ le gain proportionnel et $d_i$ le gain dérivateur. Avec une telle correction, nous allons diminué l'ordre du transfert de position/consigne et perdre le pôle en 0 menant à l'instabilité. 

\section{Choix du gain dérivateur du correcteur $C(p)$}
Passons maintenant au choix des valeurs du correcteur. On nous propose un choix particulier pour $d_i$ dans l'énoncé du TP, nous allons voir ensemble en quoi ce choix est judicieux. Nous notons, pour le procédé étudié le transfert $G(p) = \frac{N(p)}{D(p)}=\frac{N(p)}{p(1+\tau_mp)}$, la boucle fermé avec le correcteur en cours d'étude qui intervient de cette manière : 
\begin{align*}
G_{bf}(p) = \frac{Y(p)}{Y_{ref}}	&= \frac{C(p)G(p)}{1+C(p)G(p)} 
										= \frac{k_0(1+d_ip) \frac{N(p)}{D(p)}}{1+k_0(1+d_ip) \frac{N(p)}{D(p)}}	\\
									&= \frac{k_0(1+d_ip) \frac{N(p)}{p(1+\tau_mp)}}{1+ k_0(1+d_ip) \frac{N(p)}{p(1+\tau_mp)}}
\end{align*}
si l'on prend : $d_i = \tau_m$, nous pouvons retomber sur une fonction de transfert plus simple qui est :
\begin{equation}\label{eqn:boucleFermeC_PD}
G_{bf} = \frac{k_0 N(p)}{p+k_0N(p)}
\end{equation}

En sachant que N(p) contient $e^{-hp}$, nous voyons qu'avec ce correcteur, nous allons pouvoir manipuler l'influence du retard dans le système à l'aide $k_0$ et placer le pôle de la boucle fermé corrigé où nous le souhaitons.

\section{Choix du gain proportionnel du correcteur $C(p)$}
Maintenant que les calculs théoriques du correcteur ont été effectué, nous allons passer à la recherche du gain proportionnel $k_0$. Pour cela, nous allons à nous référencer aux contraintes du cahier des charges vu en Introduction. Si l'on décompose le résultat établi en \ref{eqn:boucleFermeC_PD}, il vient comme représentation de Laplace du système en boucle fermé :
\begin{equation}
G_{bf} = \frac{k_0k_rk_sk_me^{-hp}}{p+k_0k_rk_sk_me^{-hp}}
\end{equation}
Il devient donc évident que l'étude de cette boucle fermé passe par l'étude du quasi-polynôme définit par \begin{equation}\label{eqn:quasipolynome_CPD}
p+k_0k_rk_sk_me^{-hp}=0
\end{equation}

\paragraph*{Valeur du gain proportionnel}
Le cahier des charges nous impose une réponse sans oscillations : cette contrainte est rempli par l'ordre 1 de cette équation caractéristique. Pour remplir les contraintes temporelles et de dépassement, nous allons analyser la boucle fermé obtenu avec un modèle du premier ordre sous la forme : 
\begin{align*}
G(p) = \frac{K}{1+\tau p} &\text{ avec }t_r \text{ le temps de réponse } = 3.3\tau \\& \text{ et } K \text{ le gain en régime établi} 
\end{align*}
en notant tout de même que notre temps de réponse doit être établi à partir du retard du système : $t_r + h \leq 8$. Nous obtenons donc, avec une application numérique : $\tau \leq \frac{8-h}{3.3} \Leftrightarrow \tau \leq 2.42$ et $K=1$. 
Pour une identification de ces paramètres, nous prenons la fonction de transfert en boucle fermé que nous réécrivons pour correspondre avec la forme présentée précédemment : 
\begin{equation}
G_{bf} = \frac{1}{\frac{1}{k_0k_rk_sk_me^{-hp}}p+1}
\end{equation} 
Il vient donc : $\frac{1}{k_0k_rk_sk_me^{-hp}} < 2.42 \Leftrightarrow k_0 > \frac{1}{2.42k_rk_sk_me^{-hp}}$


\paragraph*{Retard Admissible}
Pour cette étude, nous allons aborder plusieurs approches. Une première, analytique, qui va consister à étudier le quasi-polynôme de la fonction de transfert en boucle fermé établi en `\ref{eqn:quasipolynome_CPD}. La seconde sera plus applicative, nous allons utiliser des tracés fréquentiels et aussi temporels pour analyser nos résultats.


Nous allons utiliser la méthode du \emph{Delay Sweeping} pour connaitre le retard admissible de notre système en boucle fermé. Pour cela,nous posons :
\begin{equation}\label{eqn:delaySweepingCorPD}
\frac{Q(j\omega)}{P(j\omega)} = \frac{k_0k_sk_mk_r}{j\omega} 
\end{equation}
On obtient alors, pour le calcul du module :
\begin{align}
\norm{\frac{Q(j\omega)}{P(j\omega)}} 	&= \norm{\frac{k_0k_sk_mk_r}{j\omega}} = 1\\
\end{align}
qui donne alors : 
\begin{align*}
\omega= k_0k_sk_mk_R
\end{align*}
Nous appliquons ensuite ce résultat sur le calcul de l'argument suivant pour pouvoir en extraire le retard maximum accessible : 
\begin{align}
&wh^* = -arg\left(-\frac{Q(j\omega)}{P(j\omega)}\right)+2\pi k,\ k \in  \mathbb{Zcircle ()}
\end{align}
qui nous donne :
\begin{align*}
h^* &= \frac{1}{\omega}arg\left(-\frac{k_0k_sk_rk_m}{j\omega}\right)\\
	&= \frac{1}{\omega}arg(-1)	- arg(j)\ \ \text{     car nous avons noté : } \omega = k_0k_sk_rk_m\\
	&= \frac{\pi}{2 \omega}
\end{align*}
\section{Équivalence avec retour d'état instantané}
Pour une loi de commande PD avec comme polynôme $Q(p) = k_1+k_2p+...+k_np^n$ dans la boucle d'asservissement, nous pouvons écrire le développement suivant : 
\begin{align*}
\frac{Y(p)}{E(p)} = \frac{G(p)}{1 + Q(p)G(p)} & \Leftrightarrow \frac{Y(p)}{E(p)} = \frac{Y(p)}{U(p) + Q(p)Y(p)}\\
&\Leftrightarrow \frac{1}{E(p)} = \frac{1}{U(p) + Q(p)Y(p)} \\
& \Leftrightarrow E(p) = U(p) + Q(p)Y(p)\\
& \Leftrightarrow U(p) = E(p) - Q(p)Y(p)
\end{align*} 
Cette dernière ligne est la caractéristique d'un retour d'état, si et seulement si les états sont disponibles sur la sortie du système.