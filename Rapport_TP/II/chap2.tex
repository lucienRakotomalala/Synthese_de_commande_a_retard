\chapter{Étude d'une commande Proportionnelle-dérivateur}
\section{Intérêt de ce correcteur}
Pour établir notre asservissement en position, nous devons faire en sorte de commander le transfert entre $u_m$ et $V_s$. Ce transfert dispose d'un intégrateur pur et d'un pôle en $-\frac{1}{\tau _m}$, qui donnent ainsi notre position instable.	
\section{Équivalence avec retour d'état instantané}
Pour une loi de commande PI avec comme polynôme $Q(p) = k_1+k_2p+...+k_np^n$ dans la boucle d'asservissement, nous pouvons écrire le développement suivant : 
\begin{align*}
\frac{Y(p)}{E(p)} = \frac{G(p)}{1 + Q(p)G(p)} & \Leftrightarrow \frac{Y(p)}{E(p)} = \frac{Y(p)}{U(p) + Q(p)Y(p)}\\
&\Leftrightarrow \frac{1}{E(p)} = \frac{1}{U(p) + Q(p)Y(p)} \\
& \Leftrightarrow E(p) = U(p) + Q(p)Y(p)\\
& \Leftrightarrow U(p) = E(p) - Q(p)Y(p)
\end{align*} 
Cette dernière ligne est la caractéristique d'un retour d'état, si et seulement si les états sont disponibles sur la sortie du système.