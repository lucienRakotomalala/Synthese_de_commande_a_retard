\chapter{Étude d'une commande Proportionnelle-dérivateur}
\section{Intérêt de ce correcteur}
Pour établir notre asservissement en position, nous devons faire en sorte de commander le transfert entre $u_m$ et $V_s$. Ce transfert dispose d'un intégrateur pur et d'un pôle en $-\frac{1}{\tau _m}$, qui donnent l'instabilité de la position du moteur à une entrée échelon. Un premier correcteur nous est proposé sous la forme :
\begin{equation}\label{eqn:correcteurProportionnel}
C(p) = k_0(1+d_ip)
\end{equation} 
avec $k_0$ le gain proportionnel et $d_i$ le gain dérivateur. Avec une telle correction, nous allons diminué l'ordre du transfert de position/consigne et perdre le pôle en 0 menant à l'instabilité. 

\section{Choix du gain dérivateur du correcteur $C(p)$}
Passons maintenant au choix des valeurs du correcteur. On nous propose un choix particulier pour $d_i$ dans l'énoncé du TP, nous allons voir ensemble en quoi ce choix est judicieux. Nous notons, pour le procédé étudié le transfert $G(p) = \frac{N(p)}{D(p)}=\frac{N(p)}{p(1+\tau_mp)}$, la boucle fermé avec le correcteur en cours d'étude qui intervient de cette manière : 
\begin{align*}
G_{bf}(p) = \frac{Y(p)}{Y_{ref}}	&= \frac{C(p)G(p)}{1+C(p)G(p)} 
										= \frac{k_0(1+d_ip) \frac{N(p)}{D(p)}}{1+k_0(1+d_ip) \frac{N(p)}{D(p)}}	\\
									&= \frac{k_0(1+d_ip) \frac{N(p)}{p(1+\tau_mp)}}{1+ k_0(1+d_ip) \frac{N(p)}{p(1+\tau_mp)}}
\end{align*}
si l'on prend : $d_i = \tau_m$, nous pouvons retomber sur une fonction de transfert plus simple qui est :
\begin{equation}\label{eqn:boucleFermeC_PD}
G_{bf} = \frac{k_0 N(p)}{p+k_0N(p)}
\end{equation}

En sachant que N(p) contient $e^{-hp}$, nous voyons qu'avec ce correcteur, nous allons pouvoir manipuler l'influence du retard dans le système à l'aide $k_0$.

\section{Choix du gain proportionnel du correcteur $C(p)$}
 
\section{Équivalence avec retour d'état instantané}
Pour une loi de commande PI avec comme polynôme $Q(p) = k_1+k_2p+...+k_np^n$ dans la boucle d'asservissement, nous pouvons écrire le développement suivant : 
\begin{align*}
\frac{Y(p)}{E(p)} = \frac{G(p)}{1 + Q(p)G(p)} & \Leftrightarrow \frac{Y(p)}{E(p)} = \frac{Y(p)}{U(p) + Q(p)Y(p)}\\
&\Leftrightarrow \frac{1}{E(p)} = \frac{1}{U(p) + Q(p)Y(p)} \\
& \Leftrightarrow E(p) = U(p) + Q(p)Y(p)\\
& \Leftrightarrow U(p) = E(p) - Q(p)Y(p)
\end{align*} 
Cette dernière ligne est la caractéristique d'un retour d'état, si et seulement si les états sont disponibles sur la sortie du système.