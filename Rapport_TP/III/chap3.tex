\chapter{Placement du spectre Fini}
Nous allons essayer de développer une loi de commande de dimension infinie permettant d'avoir un système en boucle fermé aillant un spectre fini. Pour cela, nous allons dans un premier temps définir des valeurs de pôles de façon à satisfaire le cahier des charges (voir Introduction). Ensuite, nous allons concevoir la commande de façon à avoir une boucle fermé de spectre fini et remplir le cahier des charges. Puis, nous calculerons un pré-compensateur afin de compenser l'erreur statique. Dans un quatrième temps, nous simulerons le procédé et enfin nous étudierons la robustesse de la commande.

\section{Valeurs des pôles}

\section{Commande de dimension infinie}
\section{Pré-compensation}
\section{Simulation matlab}
\section{Robustesse de la commande}
